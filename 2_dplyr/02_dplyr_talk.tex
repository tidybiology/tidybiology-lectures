\PassOptionsToPackage{unicode=true}{hyperref} % options for packages loaded elsewhere
\PassOptionsToPackage{hyphens}{url}
%
\documentclass[ignorenonframetext,]{beamer}
\usepackage{pgfpages}
\setbeamertemplate{caption}[numbered]
\setbeamertemplate{caption label separator}{: }
\setbeamercolor{caption name}{fg=normal text.fg}
\beamertemplatenavigationsymbolsempty
% Prevent slide breaks in the middle of a paragraph:
\widowpenalties 1 10000
\raggedbottom
\setbeamertemplate{part page}{
\centering
\begin{beamercolorbox}[sep=16pt,center]{part title}
  \usebeamerfont{part title}\insertpart\par
\end{beamercolorbox}
}
\setbeamertemplate{section page}{
\centering
\begin{beamercolorbox}[sep=12pt,center]{part title}
  \usebeamerfont{section title}\insertsection\par
\end{beamercolorbox}
}
\setbeamertemplate{subsection page}{
\centering
\begin{beamercolorbox}[sep=8pt,center]{part title}
  \usebeamerfont{subsection title}\insertsubsection\par
\end{beamercolorbox}
}
\AtBeginPart{
  \frame{\partpage}
}
\AtBeginSection{
  \ifbibliography
  \else
    \frame{\sectionpage}
  \fi
}
\AtBeginSubsection{
  \frame{\subsectionpage}
}
\usepackage{lmodern}
\usepackage{amssymb,amsmath}
\usepackage{ifxetex,ifluatex}
\usepackage{fixltx2e} % provides \textsubscript
\ifnum 0\ifxetex 1\fi\ifluatex 1\fi=0 % if pdftex
  \usepackage[T1]{fontenc}
  \usepackage[utf8]{inputenc}
  \usepackage{textcomp} % provides euro and other symbols
\else % if luatex or xelatex
  \usepackage{unicode-math}
  \defaultfontfeatures{Ligatures=TeX,Scale=MatchLowercase}
\fi
% use upquote if available, for straight quotes in verbatim environments
\IfFileExists{upquote.sty}{\usepackage{upquote}}{}
% use microtype if available
\IfFileExists{microtype.sty}{%
\usepackage[]{microtype}
\UseMicrotypeSet[protrusion]{basicmath} % disable protrusion for tt fonts
}{}
\IfFileExists{parskip.sty}{%
\usepackage{parskip}
}{% else
\setlength{\parindent}{0pt}
\setlength{\parskip}{6pt plus 2pt minus 1pt}
}
\usepackage{hyperref}
\hypersetup{
            pdftitle={Introduction to dplyr},
            pdfauthor={Akshay Bareja},
            pdfborder={0 0 0},
            breaklinks=true}
\urlstyle{same}  % don't use monospace font for urls
\newif\ifbibliography
\usepackage{color}
\usepackage{fancyvrb}
\newcommand{\VerbBar}{|}
\newcommand{\VERB}{\Verb[commandchars=\\\{\}]}
\DefineVerbatimEnvironment{Highlighting}{Verbatim}{commandchars=\\\{\}}
% Add ',fontsize=\small' for more characters per line
\usepackage{framed}
\definecolor{shadecolor}{RGB}{248,248,248}
\newenvironment{Shaded}{\begin{snugshade}}{\end{snugshade}}
\newcommand{\AlertTok}[1]{\textcolor[rgb]{0.94,0.16,0.16}{#1}}
\newcommand{\AnnotationTok}[1]{\textcolor[rgb]{0.56,0.35,0.01}{\textbf{\textit{#1}}}}
\newcommand{\AttributeTok}[1]{\textcolor[rgb]{0.77,0.63,0.00}{#1}}
\newcommand{\BaseNTok}[1]{\textcolor[rgb]{0.00,0.00,0.81}{#1}}
\newcommand{\BuiltInTok}[1]{#1}
\newcommand{\CharTok}[1]{\textcolor[rgb]{0.31,0.60,0.02}{#1}}
\newcommand{\CommentTok}[1]{\textcolor[rgb]{0.56,0.35,0.01}{\textit{#1}}}
\newcommand{\CommentVarTok}[1]{\textcolor[rgb]{0.56,0.35,0.01}{\textbf{\textit{#1}}}}
\newcommand{\ConstantTok}[1]{\textcolor[rgb]{0.00,0.00,0.00}{#1}}
\newcommand{\ControlFlowTok}[1]{\textcolor[rgb]{0.13,0.29,0.53}{\textbf{#1}}}
\newcommand{\DataTypeTok}[1]{\textcolor[rgb]{0.13,0.29,0.53}{#1}}
\newcommand{\DecValTok}[1]{\textcolor[rgb]{0.00,0.00,0.81}{#1}}
\newcommand{\DocumentationTok}[1]{\textcolor[rgb]{0.56,0.35,0.01}{\textbf{\textit{#1}}}}
\newcommand{\ErrorTok}[1]{\textcolor[rgb]{0.64,0.00,0.00}{\textbf{#1}}}
\newcommand{\ExtensionTok}[1]{#1}
\newcommand{\FloatTok}[1]{\textcolor[rgb]{0.00,0.00,0.81}{#1}}
\newcommand{\FunctionTok}[1]{\textcolor[rgb]{0.00,0.00,0.00}{#1}}
\newcommand{\ImportTok}[1]{#1}
\newcommand{\InformationTok}[1]{\textcolor[rgb]{0.56,0.35,0.01}{\textbf{\textit{#1}}}}
\newcommand{\KeywordTok}[1]{\textcolor[rgb]{0.13,0.29,0.53}{\textbf{#1}}}
\newcommand{\NormalTok}[1]{#1}
\newcommand{\OperatorTok}[1]{\textcolor[rgb]{0.81,0.36,0.00}{\textbf{#1}}}
\newcommand{\OtherTok}[1]{\textcolor[rgb]{0.56,0.35,0.01}{#1}}
\newcommand{\PreprocessorTok}[1]{\textcolor[rgb]{0.56,0.35,0.01}{\textit{#1}}}
\newcommand{\RegionMarkerTok}[1]{#1}
\newcommand{\SpecialCharTok}[1]{\textcolor[rgb]{0.00,0.00,0.00}{#1}}
\newcommand{\SpecialStringTok}[1]{\textcolor[rgb]{0.31,0.60,0.02}{#1}}
\newcommand{\StringTok}[1]{\textcolor[rgb]{0.31,0.60,0.02}{#1}}
\newcommand{\VariableTok}[1]{\textcolor[rgb]{0.00,0.00,0.00}{#1}}
\newcommand{\VerbatimStringTok}[1]{\textcolor[rgb]{0.31,0.60,0.02}{#1}}
\newcommand{\WarningTok}[1]{\textcolor[rgb]{0.56,0.35,0.01}{\textbf{\textit{#1}}}}
\setlength{\emergencystretch}{3em}  % prevent overfull lines
\providecommand{\tightlist}{%
  \setlength{\itemsep}{0pt}\setlength{\parskip}{0pt}}
\setcounter{secnumdepth}{0}

% set default figure placement to htbp
\makeatletter
\def\fps@figure{htbp}
\makeatother


\title{Introduction to dplyr}
\author{Akshay Bareja}
\date{10/11/2019}

\begin{document}
\frame{\titlepage}

\begin{frame}[fragile]{Loading the \texttt{proteins} and
\texttt{subcell} datasets into RStudio}
\protect\hypertarget{loading-the-proteins-and-subcell-datasets-into-rstudio}{}

The datasets can be found in the tidybiology package

To install and load the package, run the following

\begin{Shaded}
\begin{Highlighting}[]
\NormalTok{devtools}\OperatorTok{::}\KeywordTok{install_github}\NormalTok{(}\StringTok{"hirscheylab/tidybiology"}\NormalTok{)}
\KeywordTok{library}\NormalTok{(tidybiology)}
\end{Highlighting}
\end{Shaded}

To load the \texttt{proteins} and \texttt{subcell} datasets, run the
following

\begin{Shaded}
\begin{Highlighting}[]
\KeywordTok{data}\NormalTok{(proteins)}
\KeywordTok{data}\NormalTok{(subcell)}
\end{Highlighting}
\end{Shaded}

\end{frame}

\begin{frame}[fragile]{Inspecting the \texttt{proteins} dataset}
\protect\hypertarget{inspecting-the-proteins-dataset}{}

Use the \texttt{dim()} function to see how many rows (observations) and
columns (variables) there are

\begin{Shaded}
\begin{Highlighting}[]
\KeywordTok{dim}\NormalTok{(proteins)}
\end{Highlighting}
\end{Shaded}

\begin{verbatim}
## [1] 20430     8
\end{verbatim}

\end{frame}

\begin{frame}[fragile]{Inspecting the \texttt{proteins} dataset}
\protect\hypertarget{inspecting-the-proteins-dataset-1}{}

Use the \texttt{glimpse()} function to see what kinds of variables the
dataset contains

\begin{Shaded}
\begin{Highlighting}[]
\KeywordTok{glimpse}\NormalTok{(proteins)}
\end{Highlighting}
\end{Shaded}

\begin{verbatim}
## Observations: 20,430
## Variables: 8
## $ uniprot_id       <chr> "P04217", "Q9NQ94", "P01023", "A8K2U0", "U3KP...
## $ gene_name        <chr> "A1BG", "A1CF", "A2M", "A2ML1", "A3GALT2", "A...
## $ gene_name_alt    <chr> NA, "ACF ASP", "CPAMD5 FWP007", "CPAMD9", "A3...
## $ protein_name     <chr> "Alpha-1B-glycoprotein ", "APOBEC1 complement...
## $ protein_name_alt <chr> "Alpha-1-B glycoprotein)", "APOBEC1-stimulati...
## $ sequence         <chr> "MSMLVVFLLLWGVTWGPVTEAAIFYETQPSLWAESESLLKPLAN...
## $ length           <dbl> 495, 594, 1474, 1454, 340, 353, 340, 546, 672...
## $ mass             <dbl> 54254, 65202, 163291, 161107, 38754, 40499, 3...
\end{verbatim}

\end{frame}

\begin{frame}[fragile]{Basic Data Types in R}
\protect\hypertarget{basic-data-types-in-r}{}

R has 6 basic data ypes -

\textbf{character} - \texttt{"a"}, \texttt{"tidyverse"}

\textbf{numeric} - \texttt{2}, \texttt{11.5}

\textbf{integer} - \texttt{2L} (the \texttt{L} tells R to store this as
an integer)

\textbf{logical} - \texttt{TRUE}, \texttt{FALSE}

\textbf{complex} - \texttt{1+4i}

(\textbf{raw})

You will also come across the \textbf{double} datatype. It is the same
as \textbf{numeric}

\textbf{factor}. A \textbf{factor} is a collection of \emph{ordered}
character variables

\end{frame}

\begin{frame}[fragile]{Basic Data Types in R}
\protect\hypertarget{basic-data-types-in-r-1}{}

In addition to the \texttt{glimpse()} function, you can use the
\texttt{class()} function to determine the data type of a specific
column

\begin{Shaded}
\begin{Highlighting}[]
\KeywordTok{class}\NormalTok{(proteins}\OperatorTok{$}\NormalTok{length)}
\end{Highlighting}
\end{Shaded}

\begin{verbatim}
## [1] "numeric"
\end{verbatim}

\end{frame}

\begin{frame}[fragile]{(Re)Introducing \texttt{\%\textgreater{}\%}}
\protect\hypertarget{reintroducing}{}

The \texttt{\%\textgreater{}\%} operator is a way of ``chaining''
together strings of commands that make reading your code easy. The
following code chunk illustrates how \texttt{\%\textgreater{}\%} works

\begin{Shaded}
\begin{Highlighting}[]
\NormalTok{proteins }\OperatorTok\StringTok{ }
\StringTok{  }\KeywordTok{select}\NormalTok{(uniprot_id, length) }\OperatorTok\StringTok{ }
\StringTok{  }\KeywordTok{filter}\NormalTok{(length }\OperatorTok{>}\StringTok{ }\DecValTok{500}\NormalTok{) }\OperatorTok\StringTok{ }
\StringTok{  }\KeywordTok{head}\NormalTok{(}\DecValTok{1}\NormalTok{)}
\end{Highlighting}
\end{Shaded}

\begin{verbatim}
## # A tibble: 1 x 2
##   uniprot_id length
##   <chr>       <dbl>
## 1 Q9NQ94        594
\end{verbatim}

The above code chunk does the following - it takes you dataset,
\texttt{proteins}, and ``pipes'' it into \texttt{select()}

\end{frame}

\begin{frame}[fragile]{(Re)Introducing \texttt{\%\textgreater{}\%}}
\protect\hypertarget{reintroducing-1}{}

The second line selects just the columns named \texttt{uniprot\_id} and
\texttt{length} and ``pipes'' that into \texttt{filter()}. The final
line selects proteins that are longer than 500 amino acids

When you see \texttt{\%\textgreater{}\%}, think ``and then''

The alternative to using \texttt{\%\textgreater{}\%} is running the
following code

\begin{Shaded}
\begin{Highlighting}[]
\KeywordTok{filter}\NormalTok{(}\KeywordTok{select}\NormalTok{(proteins, uniprot_id, length), length }\OperatorTok{>}\StringTok{ }\DecValTok{500}\NormalTok{)}
\end{Highlighting}
\end{Shaded}

Although this is only one line as opposed to three, it's both more
difficult to write and more difficult to read

\end{frame}

\begin{frame}[fragile]{Introducing the main dplyr verbs}
\protect\hypertarget{introducing-the-main-dplyr-verbs}{}

dplyr is a package that contains a suite of functions that allow you to
easily manipulate a dataset

Some of the things you can do are -

\begin{itemize}[<+->]
\item
  select rows and columns that match specific criteria
\item
  create new variables (columns)
\item
  obtain summary statistics on individual groups within your datsets
\end{itemize}

The main verbs we will cover are \texttt{select()}, \texttt{filter()},
\texttt{arrange()}, \texttt{mutate()}, and \texttt{summarise()}. These
all combine naturally with \texttt{group\_by()} which allows you to
perform any operation ``by group''

\end{frame}

\begin{frame}[fragile]{\texttt{select()}}
\protect\hypertarget{select}{}

The \texttt{select()} verb allows you to extract specific columns from
your dataset

The most basic \texttt{select()} is one where you comma separate a list
of columns you want included

For example, if you only want to select the \texttt{uniprot\_id} and
\texttt{length} columns, run the following code chunk

\begin{Shaded}
\begin{Highlighting}[]
\NormalTok{proteins }\OperatorTok\StringTok{ }
\StringTok{  }\KeywordTok{select}\NormalTok{(uniprot_id, length) }\OperatorTok\StringTok{ }
\StringTok{  }\KeywordTok{head}\NormalTok{(}\DecValTok{1}\NormalTok{)}
\end{Highlighting}
\end{Shaded}

\begin{verbatim}
## # A tibble: 1 x 2
##   uniprot_id length
##   <chr>       <dbl>
## 1 P04217        495
\end{verbatim}

\end{frame}

\begin{frame}[fragile]{\texttt{select()}}
\protect\hypertarget{select-1}{}

If you want to select all columns \emph{except} \texttt{uniprot\_id},
run the following

\begin{Shaded}
\begin{Highlighting}[]
\NormalTok{proteins }\OperatorTok\StringTok{ }
\StringTok{  }\KeywordTok{select}\NormalTok{(}\OperatorTok{-}\NormalTok{uniprot_id) }\OperatorTok\StringTok{ }
\StringTok{  }\KeywordTok{head}\NormalTok{(}\DecValTok{1}\NormalTok{)}
\end{Highlighting}
\end{Shaded}

\begin{verbatim}
## # A tibble: 1 x 7
##   gene_name gene_name_alt protein_name protein_name_alt sequence length
##   <chr>     <chr>         <chr>        <chr>            <chr>     <dbl>
## 1 A1BG      <NA>          "Alpha-1B-g~ Alpha-1-B glyco~ MSMLVVF~    495
## # ... with 1 more variable: mass <dbl>
\end{verbatim}

\end{frame}

\begin{frame}[fragile]{\texttt{select()}}
\protect\hypertarget{select-2}{}

Finally, you can provide a range of columns to return two columns and
everything in between. For example

\begin{Shaded}
\begin{Highlighting}[]
\NormalTok{proteins }\OperatorTok\StringTok{ }
\StringTok{  }\KeywordTok{select}\NormalTok{(uniprot_id}\OperatorTok{:}\NormalTok{protein_name) }\OperatorTok\StringTok{ }
\StringTok{  }\KeywordTok{head}\NormalTok{(}\DecValTok{1}\NormalTok{)}
\end{Highlighting}
\end{Shaded}

\begin{verbatim}
## # A tibble: 1 x 4
##   uniprot_id gene_name gene_name_alt protein_name            
##   <chr>      <chr>     <chr>         <chr>                   
## 1 P04217     A1BG      <NA>          "Alpha-1B-glycoprotein "
\end{verbatim}

This code selects the following columns - \texttt{uniprot\_id},
\texttt{gene\_name}, \texttt{gene\_name\_alt}, and
\texttt{protein\_name}

\end{frame}

\begin{frame}[fragile]{\texttt{select()} exercise}
\protect\hypertarget{select-exercise}{}

Select the following columns - \texttt{uniprot\_id}, \texttt{sequence},
\texttt{length}, and \texttt{mass}

\begin{Shaded}
\begin{Highlighting}[]
\NormalTok{proteins }\OperatorTok\StringTok{ }
\StringTok{  }\KeywordTok{select}\NormalTok{(uniprot_id, sequence}\OperatorTok{:}\NormalTok{mass)}
\end{Highlighting}
\end{Shaded}

\end{frame}

\begin{frame}[fragile]{\texttt{filter()}}
\protect\hypertarget{filter}{}

The \texttt{filter()} verb allows you to choose rows based on certain
condition(s) and discard everything else

All filters are performed on some logical statement

If a row meets the condition of this statement (i.e.~is true) then it
gets chosen (or ``filtered''). All other rows are discarded

\end{frame}

\begin{frame}[fragile]{\texttt{filter()}}
\protect\hypertarget{filter-1}{}

Filtering can be performed on categorical data

\begin{Shaded}
\begin{Highlighting}[]
\NormalTok{subcell }\OperatorTok\StringTok{ }
\StringTok{  }\KeywordTok{filter}\NormalTok{(location }\OperatorTok{==}\StringTok{ "Ribosome"}\NormalTok{) }\OperatorTok\StringTok{ }
\StringTok{  }\KeywordTok{head}\NormalTok{(}\DecValTok{1}\NormalTok{)}
\end{Highlighting}
\end{Shaded}

\begin{verbatim}
## # A tibble: 1 x 5
##   ensembl_prot_id gene_name go_term    score location
##   <chr>           <chr>     <chr>      <dbl> <chr>   
## 1 18S_rRNA        18S_rRNA  GO:0005840  3.39 Ribosome
\end{verbatim}

The code chunk above only selects ribosome-associated proteins

Note that \texttt{filter()} only applies to rows, and has no effect on
columns

\end{frame}

\begin{frame}[fragile]{\texttt{filter()}}
\protect\hypertarget{filter-2}{}

Filtering can also be performed on numerical data

For example, to select proteins with a score greater than 4, run the
following code

\begin{Shaded}
\begin{Highlighting}[]
\NormalTok{subcell }\OperatorTok\StringTok{ }
\StringTok{  }\KeywordTok{filter}\NormalTok{(score }\OperatorTok{>}\StringTok{ }\DecValTok{4}\NormalTok{) }\OperatorTok\StringTok{ }
\StringTok{  }\KeywordTok{head}\NormalTok{(}\DecValTok{1}\NormalTok{)}
\end{Highlighting}
\end{Shaded}

\begin{verbatim}
## # A tibble: 1 x 5
##   ensembl_prot_id gene_name go_term    score location     
##   <chr>           <chr>     <chr>      <dbl> <chr>        
## 1 ENSP00000263100 A1BG      GO:0005576     5 Extracellular
\end{verbatim}

\end{frame}

\begin{frame}[fragile]{\texttt{filter()}}
\protect\hypertarget{filter-3}{}

To filter on multiple conditions, you can write a sequence of
\texttt{filter()} commands

For example, to select ribosome-associated proteins \emph{and} proteins
with a score greater than 4, run the following

\begin{Shaded}
\begin{Highlighting}[]
\NormalTok{subcell }\OperatorTok\StringTok{ }
\StringTok{  }\KeywordTok{filter}\NormalTok{(location }\OperatorTok{==}\StringTok{ "Ribosome"}\NormalTok{) }\OperatorTok\StringTok{ }
\StringTok{  }\KeywordTok{filter}\NormalTok{(score }\OperatorTok{>}\StringTok{ }\DecValTok{4}\NormalTok{) }\OperatorTok\StringTok{ }
\StringTok{  }\KeywordTok{head}\NormalTok{(}\DecValTok{1}\NormalTok{)}
\end{Highlighting}
\end{Shaded}

\begin{verbatim}
## # A tibble: 1 x 5
##   ensembl_prot_id gene_name go_term    score location
##   <chr>           <chr>     <chr>      <dbl> <chr>   
## 1 ENSP00000362300 AGO1      GO:0005844     5 Ribosome
\end{verbatim}

\end{frame}

\begin{frame}[fragile]{\texttt{filter()}}
\protect\hypertarget{filter-4}{}

To avoid writing multiple \texttt{filter()} commands, multiple logical
statements can be put inside a single \texttt{filter()} command,
separated by commas

\begin{Shaded}
\begin{Highlighting}[]
\NormalTok{subcell }\OperatorTok\StringTok{ }
\StringTok{  }\KeywordTok{filter}\NormalTok{(location }\OperatorTok{==}\StringTok{ "Ribosome"}\NormalTok{,}
\NormalTok{         score }\OperatorTok{>}\StringTok{ }\DecValTok{4}\NormalTok{) }\OperatorTok\StringTok{ }
\StringTok{  }\KeywordTok{head}\NormalTok{(}\DecValTok{1}\NormalTok{)}
\end{Highlighting}
\end{Shaded}

\begin{verbatim}
## # A tibble: 1 x 5
##   ensembl_prot_id gene_name go_term    score location
##   <chr>           <chr>     <chr>      <dbl> <chr>   
## 1 ENSP00000362300 AGO1      GO:0005844     5 Ribosome
\end{verbatim}

\end{frame}

\begin{frame}[fragile]{\texttt{filter()} exercise}
\protect\hypertarget{filter-exercise}{}

Filter all proteins NOT associated with the ribosome, with a score no
more than 4

\texttt{!=} = ``not equal to''

\texttt{\textless{}=} = ``less than or equal to''

\begin{Shaded}
\begin{Highlighting}[]
\NormalTok{subcell }\OperatorTok\StringTok{ }
\StringTok{  }\KeywordTok{filter}\NormalTok{(location }\OperatorTok{!=}\StringTok{ "Ribosome"}\NormalTok{,}
\NormalTok{         score }\OperatorTok{<=}\StringTok{ }\DecValTok{4}\NormalTok{)}
\end{Highlighting}
\end{Shaded}

\end{frame}

\begin{frame}[fragile]{\texttt{arrange()}}
\protect\hypertarget{arrange}{}

You can use the \texttt{arrange()} verb to sort rows

The input for arrange is one or many columns, and \texttt{arrange()}
sorts the rows in ascending order i.e.~from smallest to largest

For example, to sort rows from smallest to largest protein, run the
following

\begin{Shaded}
\begin{Highlighting}[]
\NormalTok{proteins }\OperatorTok\StringTok{ }
\StringTok{  }\KeywordTok{arrange}\NormalTok{(length) }\OperatorTok\StringTok{ }
\StringTok{  }\KeywordTok{head}\NormalTok{(}\DecValTok{3}\NormalTok{)}
\end{Highlighting}
\end{Shaded}

\begin{verbatim}
## # A tibble: 3 x 8
##   uniprot_id gene_name gene_name_alt protein_name protein_name_alt sequence
##   <chr>      <chr>     <chr>         <chr>        <chr>            <chr>   
## 1 P0DPR3     TRDD1     <NA>          T cell rece~ <NA>             EI      
## 2 P0DPI4     TRBD1     <NA>          T cell rece~ <NA>             GTGG    
## 3 P01858     <NA>      <NA>          "Phagocytos~ Tuftsin)         TKPR    
## # ... with 2 more variables: length <dbl>, mass <dbl>
\end{verbatim}

\end{frame}

\begin{frame}[fragile]{\texttt{arrange()}}
\protect\hypertarget{arrange-1}{}

To reverse this order, use the \texttt{desc()} function within
\texttt{arrange()}

\begin{Shaded}
\begin{Highlighting}[]
\NormalTok{proteins }\OperatorTok\StringTok{ }
\StringTok{  }\KeywordTok{arrange}\NormalTok{(}\KeywordTok{desc}\NormalTok{(length)) }\OperatorTok\StringTok{ }
\StringTok{  }\KeywordTok{head}\NormalTok{(}\DecValTok{3}\NormalTok{)}
\end{Highlighting}
\end{Shaded}

\begin{verbatim}
## # A tibble: 3 x 8
##   uniprot_id gene_name gene_name_alt protein_name protein_name_alt sequence
##   <chr>      <chr>     <chr>         <chr>        <chr>            <chr>   
## 1 Q8WZ42     TTN       <NA>          "Titin "     EC 2.7.11.1) (C~ MTTQAPT~
## 2 Q8WXI7     MUC16     CA125         "Mucin-16 "  MUC-16) (Ovaria~ MLKPSGL~
## 3 Q8NF91     SYNE1     C6orf98 KIAA~ "Nesprin-1 " Enaptin) (KASH ~ MATSRGA~
## # ... with 2 more variables: length <dbl>, mass <dbl>
\end{verbatim}

\end{frame}

\begin{frame}[fragile]{\texttt{arrange()} exercise}
\protect\hypertarget{arrange-exercise}{}

What happens when you apply \texttt{arrange()} to a categorical
variable?

\begin{Shaded}
\begin{Highlighting}[]
\NormalTok{proteins }\OperatorTok\StringTok{ }
\StringTok{  }\KeywordTok{arrange}\NormalTok{(gene_name_alt) }\OperatorTok\StringTok{ }
\StringTok{  }\KeywordTok{head}\NormalTok{(}\DecValTok{6}\NormalTok{)}
\end{Highlighting}
\end{Shaded}

\begin{verbatim}
## # A tibble: 6 x 8
##   uniprot_id gene_name gene_name_alt protein_name protein_name_alt sequence
##   <chr>      <chr>     <chr>         <chr>        <chr>            <chr>   
## 1 O14569     CYB561D2  101F6 LUCA12~ "Cytochrome~ EC 7.2.1.3) (Pu~ MALSAET~
## 2 P18054     ALOX12    12LO LOG12    "Arachidona~ 12S-LOX) (12S-l~ MGRYRIR~
## 3 O43715     TRIAP1    15E1.1 HSPC1~ "TP53-regul~ Protein 15E1.1)~ MNSVGEA~
## 4 O43716     GATC      15E1.2        Glutamyl-tR~ Gln) amidotrans~ MWSRLVW~
## 5 Q14596     NBR1      1A13B KIAA00~ "Next to BR~ Cell migration-~ MEPQVTL~
## 6 O14931     NCR3      1C7 LY117     "Natural cy~ Activating natu~ MAWMLLL~
## # ... with 2 more variables: length <dbl>, mass <dbl>
\end{verbatim}

\end{frame}

\begin{frame}[fragile]{\texttt{mutate()}}
\protect\hypertarget{mutate}{}

The \texttt{mutate()} verb, unlike the ones covered so far, creates new
variable(s) i.e.~new column(s). For example

\begin{Shaded}
\begin{Highlighting}[]
\NormalTok{proteins }\OperatorTok\StringTok{ }
\StringTok{  }\KeywordTok{mutate}\NormalTok{(}\DataTypeTok{sqrt_length =} \KeywordTok{sqrt}\NormalTok{(length)) }\OperatorTok\StringTok{ }
\StringTok{  }\KeywordTok{head}\NormalTok{(}\DecValTok{1}\NormalTok{)}
\end{Highlighting}
\end{Shaded}

\begin{verbatim}
## # A tibble: 1 x 9
##   uniprot_id gene_name gene_name_alt protein_name protein_name_alt sequence
##   <chr>      <chr>     <chr>         <chr>        <chr>            <chr>   
## 1 P04217     A1BG      <NA>          "Alpha-1B-g~ Alpha-1-B glyco~ MSMLVVF~
## # ... with 3 more variables: length <dbl>, mass <dbl>, sqrt_length <dbl>
\end{verbatim}

The code chunk above takes all the elements of the column
\texttt{length}, evaluates the square root of each element, and
populates a new column called \texttt{sqrt\_length} with these results

\end{frame}

\begin{frame}[fragile]{\texttt{mutate()}}
\protect\hypertarget{mutate-1}{}

Multiple columns can be used as inputs. For example

\begin{Shaded}
\begin{Highlighting}[]
\NormalTok{proteins }\OperatorTok\StringTok{ }
\StringTok{  }\KeywordTok{mutate}\NormalTok{(}\DataTypeTok{protein_length_mass =}\NormalTok{ length}\OperatorTok{/}\NormalTok{mass) }\OperatorTok\StringTok{ }
\StringTok{  }\KeywordTok{head}\NormalTok{(}\DecValTok{1}\NormalTok{)}
\end{Highlighting}
\end{Shaded}

\begin{verbatim}
## # A tibble: 1 x 9
##   uniprot_id gene_name gene_name_alt protein_name protein_name_alt sequence
##   <chr>      <chr>     <chr>         <chr>        <chr>            <chr>   
## 1 P04217     A1BG      <NA>          "Alpha-1B-g~ Alpha-1-B glyco~ MSMLVVF~
## # ... with 3 more variables: length <dbl>, mass <dbl>,
## #   protein_length_mass <dbl>
\end{verbatim}

This code takes the length of each protein and divides it by its mass

The results are stored in a new column called
\texttt{protein\_length\_mass}

\end{frame}

\begin{frame}[fragile]{\texttt{mutate()} exercise}
\protect\hypertarget{mutate-exercise}{}

Create a new column (give it any name you like) and fill it with protein
lengths divided by 100

\begin{Shaded}
\begin{Highlighting}[]
\NormalTok{proteins }\OperatorTok\StringTok{ }
\StringTok{  }\KeywordTok{mutate}\NormalTok{(}\DataTypeTok{protein_length_100 =}\NormalTok{ length}\OperatorTok{/}\DecValTok{100}\NormalTok{)}
\end{Highlighting}
\end{Shaded}

\end{frame}

\begin{frame}[fragile]{\texttt{summarise()}}
\protect\hypertarget{summarise}{}

\texttt{summarise()} produces a new dataframe that aggregates that
values of a column based on a certain condition.

For example, to calculate the mean protein length and mass, run the
following

\begin{Shaded}
\begin{Highlighting}[]
\NormalTok{proteins }\OperatorTok\StringTok{ }
\StringTok{  }\KeywordTok{summarise}\NormalTok{(}\KeywordTok{mean}\NormalTok{(length), }\KeywordTok{mean}\NormalTok{(mass))}
\end{Highlighting}
\end{Shaded}

\begin{verbatim}
## # A tibble: 1 x 2
##   `mean(length)` `mean(mass)`
##            <dbl>        <dbl>
## 1           557.       62061.
\end{verbatim}

\end{frame}

\begin{frame}[fragile]{\texttt{summarise()}}
\protect\hypertarget{summarise-1}{}

You can assign your own names by running the following

\begin{Shaded}
\begin{Highlighting}[]
\NormalTok{proteins }\OperatorTok\StringTok{ }
\StringTok{  }\KeywordTok{summarise}\NormalTok{(}\DataTypeTok{mean_length =} \KeywordTok{mean}\NormalTok{(length), }
            \DataTypeTok{mean_mass =} \KeywordTok{mean}\NormalTok{(mass))}
\end{Highlighting}
\end{Shaded}

\begin{verbatim}
## # A tibble: 1 x 2
##   mean_length mean_mass
##         <dbl>     <dbl>
## 1        557.    62061.
\end{verbatim}

\end{frame}

\begin{frame}[fragile]{\texttt{summarise()} exercise}
\protect\hypertarget{summarise-exercise}{}

Make a new table that contains the mean, median and standard deviations
of protein lengths

Use the \texttt{median()} and \texttt{sd()} functions to calculate
median and standard deviation

\begin{Shaded}
\begin{Highlighting}[]
\NormalTok{proteins }\OperatorTok\StringTok{ }
\StringTok{  }\KeywordTok{summarise}\NormalTok{(}\DataTypeTok{protein_mean =} \KeywordTok{mean}\NormalTok{(length), }
            \DataTypeTok{protein_median =} \KeywordTok{median}\NormalTok{(length),}
            \DataTypeTok{protein_sd =} \KeywordTok{sd}\NormalTok{(length))}
\end{Highlighting}
\end{Shaded}

\begin{verbatim}
## # A tibble: 1 x 3
##   protein_mean protein_median protein_sd
##          <dbl>          <dbl>      <dbl>
## 1         557.            414       596.
\end{verbatim}

\end{frame}

\begin{frame}[fragile]{\texttt{group\_by()}}
\protect\hypertarget{group_by}{}

\texttt{group\_by()} and \texttt{summarise()} can be used in combination
to summarise by groups

For example, if you'd like to know the mean location score of proteins
in each region, run the following

\begin{Shaded}
\begin{Highlighting}[]
\NormalTok{subcell }\OperatorTok\StringTok{ }
\StringTok{  }\KeywordTok{group_by}\NormalTok{(location) }\OperatorTok\StringTok{ }
\StringTok{  }\KeywordTok{summarise}\NormalTok{(}\KeywordTok{mean}\NormalTok{(score)) }
\end{Highlighting}
\end{Shaded}

\begin{verbatim}
## # A tibble: 15 x 2
##    location      `mean(score)`
##    <chr>                 <dbl>
##  1 Centrosome             4.96
##  2 Cytoplasm              4.52
##  3 ER                     2.96
##  4 Extracellular          3.55
##  5 Golgi                  4.04
##  6 Lysosome               1.65
##  7 Membrane               4.47
##  8 Mitochondria           3.12
##  9 Nucleolus              2.75
## 10 Nucleus                4.12
## 11 Other                  2.02
## 12 Peroxisome             1.52
## 13 Ribosome               1.81
## 14 Vacuole                1.69
## 15 Vesicle                2.03
\end{verbatim}

\end{frame}

\begin{frame}[fragile]{Saving a new dataset}
\protect\hypertarget{saving-a-new-dataset}{}

If you'd like to save the output of your wrangling, you will need to use
the \texttt{\textless{}-} or \texttt{-\textgreater{}} operators

\begin{Shaded}
\begin{Highlighting}[]
\NormalTok{subcell_new <-}\StringTok{ }\NormalTok{subcell }\OperatorTok\StringTok{ }
\StringTok{                 }\KeywordTok{group_by}\NormalTok{(location) }\OperatorTok\StringTok{ }
\StringTok{                 }\KeywordTok{summarise}\NormalTok{(}\KeywordTok{mean}\NormalTok{(score))  }
\end{Highlighting}
\end{Shaded}

To save \texttt{subcell\_new} as a new file (e.g.~csv)

\begin{Shaded}
\begin{Highlighting}[]
\KeywordTok{write_csv}\NormalTok{(subcell_new, }\StringTok{"subcell_new.csv"}\NormalTok{)}
\end{Highlighting}
\end{Shaded}

\end{frame}

\begin{frame}[fragile]{For more help}
\protect\hypertarget{for-more-help}{}

Run the following to access the Dplyr vignette

\begin{Shaded}
\begin{Highlighting}[]
\KeywordTok{browseVignettes}\NormalTok{(}\StringTok{"dplyr"}\NormalTok{)}
\end{Highlighting}
\end{Shaded}

\end{frame}

\end{document}
